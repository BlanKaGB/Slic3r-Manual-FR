%!TEX root = Slic3r-Manual.tex

\section{Ondulation verticale} % (fold)
\label{sec:z_wobble}
\index{Z Wobble}
\index{ondulation verticale}


Les ondulations dans les parois d'une impression peuvent être due à l'oscillation de l'axe Z. Une analyse approfondie des causes possibles est donnée par whosawhatsis\footnote{\url{http://goo.gl/iOYoK}} dans son article "Taxonomy of Z axis artifacts in extrusion-based 3d printing"\footnote{\url{http://goo.gl/ci9Gz}}, Cependant un point inportant pour les utilisateurs de Slic3r est l'oscillation provoquée par le nombre de pas de moteur qui ne correspondent pas au pas du filetage des tiges de Z. Ceci peut être résolu en vérifiant que le réglage \texttt{Layer Height} (épaisseur de couche) est un multiple de la longueur de pas complet.


La partie pertinente de l'article ci-dessus est cité ici:

\quote{Pour éviter des nervures sur le plan vertival Z, vous devriez toujours choisir une hauteur de couche qui est un multiple de la longueur de pas complet. Pour calculer la longueur de pas complet pour les vis que vous utilisez, prenez la hauteur de filet de vos vis (je recommande M6, avec un pas de 1 mm) et diviser par le nombre de pas pleins par rotation de vos moteurs (généralement 200) . Le micropas n'est pas assez precis, donc ignorez le pour ce calcul (mais en utiliser le micropas rendra le déplacement plus doux et plus silencieux). Pour les vis M6, cela fait 5 microns. C'est 4 microns pour les vis M5 utilisés par la i3, et 6,25 microns pour les vis M8 utilisés par la plupart des autres repraps.  Une hauteur de couche de 200 microns (0,2 mm), par exemple, fonctionnera parfaitement sur l'une de ces vis, car 200 = 6,25 * 32 = 5 * 40 = 4 * 50.}


% section z_wobble (end)
