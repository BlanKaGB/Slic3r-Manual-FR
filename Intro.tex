%!TEX root = Slic3r-Manual.tex

\section{Pr\'esentation} % (fold)
\label{sec:overview}

Slic3r est un outil qui traduit des mod\`eles 3D en instructions interpr\'et\'ees par une imprimante 3D. Il d\'ecoupe le mod\`ele en couches horizontales et g\'en\`ere les chemins appropri\'es pour les combler.

Slic3r est inclus dans plusieurs logiciels: Pronterface, Repetier-host, ReplicatorG, et peut \^etre utilis\'e comme un programme autonome.

Ce manuel fournira des conseils sur la fa\c{c}on d'installer, configurer et utiliser Slic3r afin de produire d'excellentes impressions.

% section overview (end)


\section{Buts \& Philosophie} % (fold)
\label{sec:goals_philosophy}

Slic3r est un projet original commenc\'e en 2011 par Alessandro Ranellucci (alias Sound), qui a utilis\'e sa connaissance approfondie du langage Perl pour cr\'eer une application rapide et facile \`a utiliser. La lisibilit\'e et la maintenabilit\'e du code font partis les objectifs de conception.

Le programme est constamment en cours d'am\'elioration, Alessandro et les autres contributeurs du projet, apportent r\'eguli\`erement de nouvelles fonctionnalit\'es et les corrections de bogues.

% section goals_philosophy (end)


\section{Faire un don} % (fold)
\label{sec:donating}

Slic3r a commenc\'e comme un travail d'un seul homme, d\'evelopp\'e exclusivement par Alessandro \`a ses heures perdues, en tant que d\'eveloppeur ind\'ependant, ce qui a un co\^ut direct pour lui. En lib\'erant g\'en\'ereusement Slic3r au public en tant que logiciel libre , sous licence GPL, il a permis \`a beaucoup de profiter de son travail.

Il est possible de contribuer par un don. Vous trouverez plus de d\'etails \`a l'adresse: \url{http://slic3r.org/donations}.

% section donating (end)