%!TEX root = Slic3r-Manual.tex
\index{download}
\index{telechargement}
\index{binaries}
\index{binaires}
\index{Source Code}
\index{Code Source}
\index{GitHub}
\index{license}
\index{licence}

\fbox{
	\parbox{\linewidth}{
		Slic3r est un logiciel libre, sous licence GNU Affero General Public License, version 3.
	}
}	

\section{T\'el\'echargement}

\subsection{Slic3r} % (fold)
\label{sub:slic3r}
Slic3r peut \^etre t\'el\'echarg\'es directement \`a partir de: \url{http://slic3r.org/download}.

Des paquets pr\'e-compil\'es sont disponibles pour Windows, Mac OS X et Linux. Les utilisateurs de Windows et Linux peuvent choisir entre 32 et 64 bit versions pour correspondre \`a leur syst\`eme.
% subsection slic3r (end)

\subsection{Manuel} % (fold)
\label{sub:manual}

La derni\`ere version de ce document en anglais, avec le code source {\LaTeX}, peut \'etre trouv\'e sur: \url{https://github.com/alexrj/Slic3r-Manual}

La derni\`ere version de ce document en fran\c{c}ais, avec le code source {\LaTeX}, peut \'etre trouv\'e sur: \url{https://github.com/llegoff/Slic3r-Manual-FR}

% subsection manual (end)

\subsection{Code Source} % (fold)
\label{sub:source}

Le code source est disponible via GitHub: \url{https://github.com/alexrj/Slic3r}. Pour plus de d\'etail sur la compilation depuis le code source, voir §\ref{sec:building_from_source} plus bas.

% subsection source (end)

\section{Installer}

\subsection{Windows}

D\'ecompressez le fichier zip t\'el\'echarg\'e, dans un dossier de votre choix, il n'y a pas de script d'installation. Le dossier r\'esultant contient deux ex\'ecutables:
\begin{itemize}
\item \texttt{slic3r.exe} - d\'emarre la version interface graphique.
\item \texttt{slic3r-console.exe} - peut \^etre utilis\'e \`a partir de la ligne de commande.
\end{itemize}

Le fichier zip peut alors \^etre supprim\'e.

\subsection{Mac OS X}

Double-cliquez sur le fichier dmg t\'el\'echarg\'e, une instance de Finder devrait s'ouvrir avec une ic\^one du programme Slic3r. Acc\'edez au r\'epertoire Applications et faites glisser y l'ic\^one Slic3r.
Le fichier dmg peut alors \^etre supprim\'e.

\subsection{Linux}

Extraire l'archive dans un dossier de votre choix.
soit:
\begin{itemize}
\item Lancer Slic3r directement par l'ex\'ecutable Slic3r, trouv\'e dans le r\'epertoire bin, ou
\item Installez Slic3r en ex\'ecutant le fichier ex\'ecutable do-install, \'egalement trouv\'e dans le dossier bin.
\end{itemize}
Le fichier d'archive peut alors \^etre supprim\'e.



\section{Compiler depuis le code source} % (fold)
\label{sec:building_from_source}

Pour les plus t\'em\'eraires, Slic3r peut \^etre compil\'e \`a partir des derni\`ers fichiers sources trouv\'ees sur GitHub\footnote{\url{https://github.com/alexrj/Slic3r}}.

Les instructions les plus r\'ecentes pour la compilation des fichiers sources et l'ex\'ecution peuvent \^etre trouv\'es sur le wiki Slic3r.

\begin{itemize}
    \item \textbf{GNU Linux} \par\url{https://github.com/alexrj/Slic3r/wiki/Running-Slic3r-from-git-on-GNU-Linux}
    \item \textbf{OS X} \par\url{https://github.com/alexrj/Slic3r/wiki/Running-Slic3r-from-git-on-OS-X}
    \item \textbf{Windows} \par\url{https://github.com/alexrj/Slic3r/wiki/Running-Slic3r-from-git-on-Windows}

\end{itemize}
